\documentclass[11pt]{article}

\usepackage{hyperref}
\usepackage{amssymb}
\usepackage[letterpaper]{geometry}

\setlength{\parindent}{0em}
\setlength{\parskip}{1em}

\begin{document}

\title{UW REALab JuMP tutorial}
\author{Gord Stephen}
\date{February 27, 2020}
\maketitle

\section{A Basic Linear Model}

Let's start with a very simple model that chooses a least-cost set of potential wind and solar PV installations to supply a predetermined hourly demand profile for a hypothetical power system. (This is theoretically equivalent to the long-run market equilibrium for the system, under perfect competition assumptions).

Since the model is linear (for now), we'll allow fractions of an installation to be built. Any demand that is unserved will incur a penalty corresponding to the system's value of lost load (VoLL). We'll model one year of investment decisions and subsequent operations, and represent annualized capital costs accordingly.

\subsection{Problem Parameters}

$C^{pv}$, the annualized fixed cost of PV capacity (\$/MW)

$P^{pv}_i$, the nameplate capacity of PV site $i$ (MW)

$G^{pv}_{it}$, the available capacity of PV site $i$ at time $t$ (MW)


$C^{wind}$, the annualized fixed cost of wind capacity (\$/MW)

$P^{wind}_i$, the nameplate capacity of wind site $i$ (MW)

$G^{wind}_{it}$, the available capacity of wind site $i$ at time $t$

$L_t$, system load at time $t$ (MW)

$V$, system VoLL (\$/MWh)

\subsection{Investment Decisions}

$b^{wind}_i$, the fraction of wind site $i$ to build

$b^{pv}_i$, the fraction of PV site $i$ to build

\subsection{Operations Decisions}

$g^{wind}_{t}$, wind dispatch at time $t$

$g^{pv}_{t}$, PV dispatch at time $t$

$r_t$, dropped load at time $t$

\subsection{Objective Function}

Minimize total system cost (fixed costs plus unserved energy penalties):

\begin{equation}
C^{wind} (P^{wind})^T b^{wind} + C^{PV} (P^{PV})^T b^{pv} + V \mathbf{1}^T r
\end{equation}

\subsection{Investment Constraints}

Build decisions must be a fraction of the site:

\begin{equation}
    0 \leq b^{wind} \leq 1
\end{equation}

\begin{equation}
    0 \leq b^{pv} \leq 1	
\end{equation}

\subsection{Operations Constraints}

Power must balance (or go unserved) in each time period:

\begin{equation}
	L = g^{wind} + g^{pv} + r
\end{equation}

Unserved energy cannot be negative:

\begin{equation}
    0 \leq r	
\end{equation}

Wind and solar generation cannot be negative or exceed built capacity available in that hour:

\begin{equation}
    0 \leq g^{wind} \leq (G^{wind})^T b^{wind}
\end{equation}

\begin{equation}
    0 \leq g^{pv} \leq (G^{pv})^T b^{pv}
\end{equation}

\textit{\textbf{Exercise:} Code up this linear program, solve it, and check the results. How much energy is unserved, what is the corresponding cost to the system, and how does that compare to system cost overall?}

\section{Adding Binary Variable Constraints}

Right now the wind and solar sites are able to be built out to some fraction of the total size. Since some of the costs of building the project are fixed (grid connection costs, etc), this may result in underestimated system costs overall. To address this, constrain the $b^{wind}$ and $b^{pv}$ variables to be binary (either 0 or 1, no intermediate values allowed). Note that this makes the problem non-convex.

\textit{\textbf{Exercise:} Replace the existing build bounds with a binary constraint and compare the model results. What is the impact on investment decisions and total system cost?}

\section{Adding SOS-1 Constraints}

Site-by-site build decisions provide the model with maximum flexibility to exploit diversity in resource availability. However, this also means the model has an exponentially-growing number of discrete options to consider ($2^n$ for $n$ sites). If resource generation is correlated anyways, this flexibility may not provide much extra value.

\textit{\textbf{Exercise:} Check the correlation matrices for both the wind generation data and the solar generation data. Could we be justified in treating site choices within a given resource type as economic substitutes rather than complements?}

If we do take wind (or solar) sites as substitutes for one another, we can consider sorting those sites according to some measure of resource quality or value-for money, and building them in that order. A promising metric to sort on is the site's levelized cost of energy, or LCOE. Note that since we are assuming a constant \$/MW cost within each resource type, LCOE is also inversely proportional to capacity factor, which is slightly easier to calculate. We can redefine our generator data as follows:

$P^{pv}_i$, the cumulative nameplate capacity of the first $i$ PV sites in the build order (MW)

$G^{pv}_{it}$, the cumulative available capacity of the first $i$ PV sites in the build order at time $t$ (MW)

$P^{wind}_i$, the cumulative nameplate capacity of the first $i$ wind sites in the build order (MW)

$G^{wind}_{it}$, the cumulative available capacity of the first $i$ wind sites at time $t$

Under this new definition, the $b^{wind}$ and $b^{pv}$ sets of variables need to be restricted so that only one element of each can take the value 1. This is a special case of a special order set 1 (SOS-1) constraint, which allows only one variable in a set of variables to be non-zero.

Like the binary constraints in the previous section, SOS-1 constraints are non-convex, but modern commercial optimization solvers can still deal with them reasonably well. Although we won't apply them in this way here, SOS-1 constraints can be very useful in expressing complementary slackness conditions in equilibrium models.

Note that beyond these changes, the rest of our optimization formulation can be left unchanged: the only difference is that there are now $n$ build options (instead of $2^n$) for a resource type with $n$ sites.

\textit{\textbf{Exercise:} Implement these changes to the problem data and add the SOS-1 constraints. How do generator investments change as a result?}

\section{Adding Integer Variable Constraints}

You may have noticed by now that a 100\% wind and solar power system seems rather expensive (in terms of both capital investments and opportunity costs of unserved energy) given the provided technology choices and cost data. (Storage technologies can reduce these system costs substantially, but are slightly more complicated to model, so we're ignoring them here.) Let's give the model some thermal generation options and see how the equilibrium generation portfolio changes. This will require augmenting our model formulation.

\subsection{New Problem Parameters}

$G^{thermal}_i$, the maximum generating capacity of thermal unit type $i$ (MW)

$C^{thermal}_i$, the annualized fixed cost of capacity for thermal unit type $i$ (\$/MW)

$M^{thermal}_i$, the marginal cost of generation for thermal unit type $i$ (\$/MWh)

\subsection{New Decision Variables}

$b^{thermal}_i$, the number of thermal units of type $i$ to build. This value will be restricted to non-negative integers (a half-built coal plant isn't all that useful).

$g^{thermal}_{it}$, the power generated by thermal unit type $i$ in time $t$ (MW)

\subsection{New Objective Function Terms}

We need to add thermal unit fixed and variable costs to our existing objective function:

\begin{equation}
	\sum_i C^{thermal}_i G^{thermal}_i b^{thermal}_i + (M^{thermal})^T g^{thermal} \mathbf{1} 
\end{equation}

\subsection{New Investment Constraints}

Thermal units built must be non-negative integers:

\begin{equation}
	b^{thermal}_i \in \mathbb{Z}_+
\end{equation}

\subsection{New Operations Constraints}

The power balance should be updated to include thermal generation:

\begin{equation}
	L = (g^{thermal})^T \textbf{1} + g^{wind} + g^{pv} + r
\end{equation}

Thermal generation for each unit type should be between zero and the installed capacity:

\begin{equation}
	0 \leq g^{thermal}_{it} \leq b^{thermal}_i G^{thermal}_i
\end{equation}

\textit{\textbf{Exercise:} Implement these changes and solve the new model. How are total system costs and investment levels in each technology affected?}

\section{Resolving for Dual Variables}

Now that our system includes technologies with non-zero marginal costs, hourly electricity prices are starting to get interesting. Unfortunately, we can't directly use our solution to the previous model to check the dual variables associated with our power balance constraint, since that problem wasn't convex.

\textit{\textbf{Exercise:} Set the investment decision variables to be fixed to their values from the previous section's results. You can now remove the non-convex constraints (binary, integer, SOS-1) from those variables without changing the optimal solution: you've essentially reduced the problem to a linear operations-only model. Re-solve this version of the model and calculate the system's average hourly electricity price.}

\section{Next Steps}
Congratulations, you now have a simple but powerful model for studying power system investment and operations! There are many ways to extend this model to study interesting techo-economic questions. For example:

\begin{itemize}

\item Implement clustered unit commitment (see \url{https://ieeexplore.ieee.org/abstract/document/7350236}) to study system flexibility and renewables curtailment issues.

\item Allow the model to build and operate energy storage. What is the impact on investment levels, energy prices and renewables penetration at economic equilibrium?

\item Allow for price-responsive (elastic) demand. How do system costs change with increasing price elasticity? \textit{Implementation hint: This requires a switch from minimizing system costs to maximizing global economic welfare. The notion of ``unserved energy'' goes away. Consider the energy market's supply-demand curves in each hour - if the market's demand curve is linear, the problem becomes a quadratic program.}

\item Extend the operations market to provision ancillary services such as reserve products or inertia in addition as energy. How do these new economic signals impact investments?

\item Add a demand curve to provision capacity or resource adequacy outside of the hourly market. How do these new economic signals impact investments?
\textit{Implementation hint: The structure of this demand curve is very similar to the implementation for price-responsive hourly electricity demand.}

\item Implement a renewable portfolio standard (RPS) constraint. What is the impact on system cost, and what is the resulting clearing price for renewable energy credits? 

\item Model a carbon pricing policy by appropriately adjusting the marginal costs for fossil fuel generation. What is the impact on energy prices and technology generation shares?

\item Model a cap-and-trade system by imposing an upper bound on overall generator carbon emissions. What is the impact on system cost, and how does the emissions permit clearing price change as a function of the emissions cap?

\item Add a transmission network to consider the impact of congestion and locational marginal prices. Depending on your research question, you could use a coarser multi-region network with pipe-and-bubble interface constraints, or a full-blown nodal model with DC-OPF.

\item Extend the temporal extent of the model to consider multiple years of operations and when to start building a generator, taking into account the duration of permitting and construction process, and life span of other assets. You can also consider allowing generators to retire early if their fixed costs no longer outweigh their value to the system.

\end{itemize}

\end{document}